\documentclass[12pt,a4paper]{extarticle}

\usepackage[utf8]{inputenc}
\usepackage{amsmath}
\usepackage{ragged2e}
\usepackage{varwidth}
\usepackage{indentfirst}
\usepackage{calc}
\usepackage{tabularx}

\usepackage[small]{titlesec}
\usepackage[a4paper,
    left=30mm,
    right=20mm,
    top=20mm,
    bottom=20mm]{geometry}

\justifying
\sloppy

\setlength{\abovecaptionskip}{\baselineskip + \abovecaptionskip}

\usepackage{fontspec}

\defaultfontfeatures{Ligatures={TeX},Renderer=Basic}
\setmainfont[Ligatures={TeX,Historic}]{Times New Roman}
\setlength{\parindent}{1.25cm}
\linespread{1}

% hyperlinks
\usepackage[final,hidelinks]{hyperref}

\renewcommand{\UrlFont}{}

% resourses list
\usepackage[square,numbers,sort&compress]{natbib}

\bibliographystyle{belarus-specific-utf8gost780u}
\def\BibUrlRus#1{-- Режим доступа~: \url{#1}}
\def\BibDateRus#1{-- Дата доступа~: \parsedate{#1}}
\def\BibUrl#1{\BibUrlRus{#1}}
\def\BibDate#1{\BibDateRus{#1}}
\def\BibTitleFormat#1{\StrSubstitute{#1}{ - }{ --- }}

% page numbering
\usepackage{fancyhdr}

\pagestyle{fancy}
\fancyhf{}
\fancyfoot[R]{\thepage}
\renewcommand{\headrulewidth}{0pt}
\addtocontents{toc}{\protect\thispagestyle{fancy}}

% lists
\usepackage{enumitem}

\setlist{
    topsep=0em,
    itemsep=0em,
    parsep=0em,
    listparindent=\parindent
}

\setlist[itemize,0]{itemindent=\parindent + 2.2ex, leftmargin=0ex, label=--}
\setlist[enumerate,1]{itemindent=\parindent + 2.7ex, leftmargin=0ex}
\setlist[enumerate,2]{itemindent=\parindent + \parindent - 2.7ex}
\renewcommand{\labelitemi}{\textendash}

% code listing
\usepackage{listings}
\usepackage[dvipsnames]{xcolor}

\lstset{
    columns=fullflexible,
    basicstyle=\ttfamily\small,
    tabsize=4,
    breaklines=true,
    breakatwhitespace=false,
    showstringspaces=false,
    language=Java,
    linewidth=\textwidth,
    commentstyle=\color{gray},
    keywordstyle=\color{orange},
    stringstyle=\color{Green},
}

% sections and toc
\newcommand{\sectionuppercase}[1]{\section[#1]{\uppercase{#1}}}

\makeatletter
\renewcommand{\l@section}{\@dottedtocline{1}{0.5em}{1.2em}}
\renewcommand{\l@subsection}{\@dottedtocline{2}{1.7em}{2.0em}}
\makeatother

% images
\usepackage{graphicx}
\usepackage{float}

\graphicspath{ {images/} }

% images and tables caption
\usepackage{caption}
\usepackage{subcaption}

\DeclareCaptionLabelFormat{stbfigure}{Рисунок #2}
\DeclareCaptionLabelFormat{stbtable}{Таблица #2}
\DeclareCaptionLabelSeparator{stb}{~--~}
\captionsetup{labelsep=stb}
\captionsetup[figure]{labelformat=stbfigure,justification=centering}
\captionsetup[table]{labelformat=stbtable,justification=raggedright}
\renewcommand{\thesubfigure}{\asbuk{subfigure}}

\setlist[enumerate,1]{itemindent=2.7ex,leftmargin=0ex}
\setlist[enumerate,2]{itemindent=0em}

\renewcommand\tabularxcolumn[1]{m{#1}}

\begin{document}

\newpage

\thispagestyle{empty}

\raggedright

\begin{center}
    Учреждение образования \\
    «БЕЛОРУССКИЙ ГОСУДАРСТВЕННЫЙ УНИВЕРСИТЕТ \\ ИНФОРМАТИКИ И РАДИОЭЛЕКТРОНИКИ» \\
    Факультет компьютерных систем и сетей \\
    Кафедра информатики \\
\end{center}

\hfill\begin{varwidth}{\textwidth}
    УТВЕРЖДАЮ \\
    Заведующий кафедрой информатики \\
    \_\_\_\_\_\_\_\_\_\_\_\_\_\_Сиротко С.И. \\
    «\_\_\_\_» \_\_\_\_\_\_\_\_\_\_\_\_\_ 2025 г.\\
\end{varwidth}

\begin{center}
    \textbf{
        ЗАДАНИЕ \\
        по курсовой работе \\
    }
    \hfill Группа 453501 \\
    Студенту Скроботу Денису Алексеевичу
\end{center}

\begin{enumerate}[label=\textbf{\arabic*}.]
    \item \textbf{Тема работы:} Игра "Pixel Wars"
    \item \textbf{Сроки сдачи студентом законченной работы:} 06.06.2025
    \item \textbf{Исходные данные к проекту:} Для написания курсового проекта был выбран язык программирования Java и фреймворк Spring.
    \item \textbf{Содержание расчетно-пояснительной записки} (перечень подлежащих разработке вопросов):
          \begingroup
          \\ Введение
          \\ Раздел 1. Обзор существующих аналогов
          \\ Раздел 2. Архитектура приложения
          \\ Раздел 3. Технологии программирования, используемые для решения поставленных задач
          \\ Раздел 4. Технические особенности реализации
          \\ Раздел 5. Тестирование
          \\ Заключение
          \\ Список использованных источников
          \\ Приложение А. Листинг программного кода
          \endgroup
    \item \textbf{Консультанты по работе:} Крупенич Е.Г.
    \item \textbf{Дата выдачи задания:} 28.02.2025 г.
    \item \textbf{Календарный график работы над проектом на весь период проектирования} (с указанием сроков выполнения и трудоемкости отдельных этапов):
\end{enumerate}

\vspace{\baselineskip}

\centering
\begin{tabularx}{\columnwidth}{| >{\centering\hsize=3ex}X | >{\centering\hsize=35ex\arraybackslash}X | >{\centering\arraybackslash}X | >{\centering\hsize=12ex\arraybackslash}X |}
    \hline
    № п/п & Наименование этапов курсового проекта      & Срок выполнения этапов проекта & Примечание       \\
    \hline
    1.    & Поиск и написание теоретической информации & 01.03.2025                     & 30\%             \\
    \hline
    2.    & Написание и отладка программы              & 01.04.2025                     & 60\%             \\
    \hline
    3.    & Демонстрация, заключение                   & 01.05.2025                     & 100\%            \\
    \hline
    4.    & Защита курсовой работы                     & 06.06.2025                     & Согласно графику \\
    \hline
\end{tabularx}

\raggedright

\vspace{\baselineskip}

\begin{varwidth}{\textwidth}
    Руководитель \hspace{5ex} \_\_\_\_\_\_\_\_\_\_\_\_\_\_\_\_\_\_\_\_\_\_\_\_\_\_\_ (Крупенич Е.Г.) \\
    Задание принял к исполнению 28.02.2025 \hfill \_\_\_\_\_\_\_\_\_\_\_\_\_\_\_ \\
    (\_\_\_\_\_\_\_\_\_\_\_\_\_\_\_\_\_\_\_\_\_\_\_) \\
\end{varwidth}

\end{document}